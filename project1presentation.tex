\documentclass[aspectratio=169]{beamer}
% if you need to pass options to natbib, use, e.g.:
% \PassOptionsToPackage{numbers, compress}{natbib}
% before loading nips_2016
%
% to avoid loading the natbib package, add option nonatbib:
% \usepackage[nonatbib]{nips_2016}

%\usepackage{nips_2016}

% to compile a camera-ready version, add the [final] option, e.g.:
%\usepackage[final,nonatbib]{nips_2016}

%\usepackage[utf8]{inputenc} % allow utf-8 input
\usepackage[T1]{fontenc}    % use 8-bit T1 fonts
\usepackage{hyperref}       % hyperlinks
\usepackage{url}            % simple URL typesetting
\usepackage{booktabs}       % professional-quality tables
\usepackage{amsfonts}       % blackboard math symbols
\usepackage{nicefrac}       % compact symbols for 1/2, etc.
\usepackage{microtype}      % microtypography
%\usepackage[shortlabels]{enumitem}
\usepackage{color}
\usepackage{ragged2e}
\usepackage[square,sort,comma,numbers]{natbib}
\usepackage[utf8]{inputenc}
%\usepackage[round]{natbib}
\bibliographystyle{apalike}

\usepackage[T1]{fontenc}
\usepackage{lmodern}

\usepackage{amsmath,amsthm,amssymb,amsfonts}
%\newtheorem{theorem}{Theorem}
%\newtheorem{proposition}{Proposition}
%\newtheorem{corollary}{Corollary}

\newcommand{\EE}{\mathbb{E}}
\newcommand{\PP}{\mathbb{P}}
\newcommand{\Ss}{\mathcal{S}}
\newcommand{\Ii}{\mathcal{I}}
\newcommand{\Ee}{\mathcal{E}}
\newcommand{\Rr}{\mathcal{R}}
\DeclareMathOperator*{\argmin}{arg\,min}
\DeclareMathOperator*{\argmax}{arg\,max}

%\title{Bayes-Optimal Intrinsically Motivated Learning v.s. Empowerment Maximisation}
\title{Deriving Intrinsic Motivation from \\ Uncertainty about Future Goals}
\subtitle{https://github.com/jnegrea/csc2541project1}
\author{
 Jeffrey Negrea
}
\institute{
 Department of Statistical Sciences\\
 University of Toronto\\
 \texttt{negrea@utstat.toronto.edu}}
 
\usetheme{default}
\usecolortheme{seahorse} 

\AtBeginSection[]
{
  \begin{frame}
    \frametitle{Table of Contents}
    \tableofcontents[currentsection]
  \end{frame}
}

\setbeamertemplate{itemize items}[default]
\setbeamertemplate{enumerate items}[default]

\begin{document}
\frame{\titlepage}

\begin{frame}
	\frametitle{Table of Contents}
	\tableofcontents
\end{frame}

\section{What Is Intrinsically Motivated Reinforcement Learning?}
\begin{frame}
	\frametitle{Heuristics of Reinforcement Learning}
	\begin{itemize}
		\item Recall that \textit{reinforcement learning} addresses the problem of how machines can learn to perform a task by trial and error.\pause\vspace{0.5em}
		\begin{itemize}
			\item A reward mechanism is pre-prescribed for the learner 
			\item The learner does not know the topology of the state space
			\item The learner does not know the reward of each state	
			\item The learner does not know the transition mechanics\pause\vspace{0.5em}
		\end{itemize}
		\item The machine learns the topology of the state space, the transition mechanics, the topography of the reward function, and how to use the mechanics to optimise the specified reward.
	\end{itemize}
\end{frame}

\begin{frame}
	\frametitle{Questions Leading to Intrinsic Motivation}
	\begin{itemize}
		\item What if the reward function has yet to be specified?\pause\vspace{1em}
		\item For animal learners, reward signals are recieved by engaging in exploratory, `fun' behaviours. Can machines be incited to play and explore? To have fun?\pause\vspace{1em}
		\item Exploratory behaviour is important for animal development -- human children learn about the world by playing. Can a machine which learns by playing outperform one that does not when performance is integrated over a wide variety of tasks?
	\end{itemize}
\end{frame}

\section{Qualitative Aspects of Intrinsic Motivation}
\begin{frame}
	\frametitle{Qualitative Aspects of Intrinsic Motivation I}
	\begin{itemize}
		\item In order to teach a machine to `play' we need measures which quantify what it means to `play' effectively. \vspace{.5em}
		\item In order to develop such measures we need heuristics which defines what it means to `play effectively.' \vspace{0.5em} \pause
		\item \citet{schmidhuber2010formal} posits that an intrinsically motivated learner should have:
		\begin{itemize}
			\item An adaptive world model
			\item A learning algorithm to update the model
			\item An intrinsic reward system measuring model improvement
			\item A behavioural policy optimising intrinsic reward\pause
		\end{itemize}\vspace{0.5em}
		\item The kernel of this thesis is that intrinsic motivation is derived from the pursuit of a better model of the dynamics of the world. 
	\end{itemize}
\end{frame}

\begin{frame}
	\frametitle{Qualitative Aspects of Intrinsic Motivation II}
	\begin{itemize}
		\item \citet{salge2014empowerment} posit that an intrinsically motivated learner should seek states of high (heuristic) \textit{empowerment}.
		\item Empowerment is an intrinsic utility measure which is \textit{Local, Universal, and Task-Independent}:\pause\vspace{.5em}
		\begin{itemize}
			\item \textit{Local}: An agent can determine intrinsic utility of a state from the local environment\pause\vspace{.5em}
			\item \textit{Universal}: The utility scale should be independent of the structure of the actor and environment -- utility should be comparable across distinct problem classes\pause\vspace{.5em}
			\item \textit{Task-Independent}: The utility is independent of any particular goal or reward function
		\end{itemize}
	\end{itemize}
\end{frame}

\section{Quantitative Measures for Intrinsic Motivation}
\begin{frame}
	\frametitle{Quantitative Measures for Intrinsic Motivation I: Optimal Reward}
	\begin{itemize}
		\item \citet{singh2010intrinsically} introduces the concept of \textit{optimal reward functions}.
		\item Requires a fixed fitness function to be pre assigned, as in traditional RL\pause
		\item The learning problem is posed as a two-stage optimisation problem;\pause
		\begin{itemize}
			\item Find the reward function such that an RL agent maximising the reward has highest expected fitness\pause
			\item Find the RL strategy which maximises the optimal reward function
		\end{itemize}\pause		 
		\item By construction, performs no worse on average than RL using the the natural reward corresponding to fitness. 
		\item Avoids greedy behaviour in favour of exploratory behaviour (when beneficial).
	\end{itemize}
\end{frame}

\begin{frame}
	\frametitle{Quantitative Measures for Intrinsic Motivation II: Empowerment}
	\begin{itemize}
		\item \textit{Empowerment} (metric) aims to achieve the heuristic it is named for
		\item Quantified as the channel capacity of a state: 
	\end{itemize}
	\[\Ee(s)=\max_{\omega\in\Omega_s}\Ii(A,S_1|s)=\max_{\omega\in\Omega_s}\EE\left[\log\frac{p(S_1,A|s)}{\omega(A|s)p(S_1|s)}\right]=\max_{\omega\in\Omega_s}\EE\left[\log\frac{p(S_1|s,A)}{p(S_1|s)}\right] \]
	{\small Where the expectation is taken with respect to the joint distribution of the action and the resultant state, $(A,S_1)$, conditional on the starting state, $s$, for a fixed choice of $\omega$ --- the distribution of the action given the starting state.} \pause
	\begin{itemize}
		\item Discussed thoroughly in \citet{singh2010intrinsically}, applied in \citet{mohamed2015variational}
		\item Aims to find the state in which an actor is most able to travel to any arbitrary state\pause\vspace{0.5em}
		\item Not obvious if the information theoretic definition is suitable
		\item Does not consider the how rewards may be assigned in the future
	\end{itemize}
\end{frame}

\begin{frame}
	\frametitle{Quantitative Measures for Intrinsic Motivation III: Expected Reward}
	\begin{itemize}
		\item Purpose of my project: Define a Bayesian framework for intrinsically motivation.
		\item Assume the agent has an internal prior on the class of possible reward functions\pause
		\item Determine the policy which maximises the \textit{intrinsic expected reward} (iER)
		\begin{itemize}
			\item Expectation is taken with respect to the random reward function, possibly random choice of action, and the results of actions (the latter two may form a feedback loop)\pause
			\item \textit{Intrinsic} since the prior on potential rewards is internal to the agent\pause\vspace{0.5em}
			\item Agent learns about the environment and transition mechanics as in traditional RL \textit{and} 
			\item Agent learns about distribution of rewards and how to improve heuristic empowerment by computing the posterior distribution of rewards and updating the iER.\pause			
		\end{itemize}		 
		\item Originally motivated by goal to show that the empowerment metric was Bayes-optimal the class of problems in \citet{mohamed2015variational}
		\begin{itemize}
			\item It turns out this is false
		\end{itemize}
	\end{itemize}
\end{frame}

\begin{frame}[allowframebreaks]
\frametitle{References}
\nocite{*}
\bibliography{project1}
\end{frame}
\end{document}
